\documentclass{sig-alternate}
\usepackage[utf8]{inputenc}
\usepackage[T1]{fontenc}

\newcommand{\superscript}[1]{\ensuremath{^{\textrm{#1}}}}
\def\sharedaffiliation{\end{tabular}\newline\begin{tabular}{c}}
\def\wu{\superscript{*}}
\def\wg{\superscript{\dag}}

\usepackage{listings} 

% Typography
\usepackage{times}
\usepackage{mathptmx}
\usepackage{microtype}
\usepackage[normalem]{ulem}
\usepackage[pdftex,bookmarks,bookmarksopen,bookmarksdepth=2,
            urlcolor=black,colorlinks=true,linkcolor=black,citecolor=black]{hyperref}
\usepackage[capitalize,noabbrev]{cleveref}
\def\UrlFont{\em}

% Graphics
\usepackage{tikz}
\usepackage{tkz-graph}
\usetikzlibrary{arrows,positioning,shapes.misc}
\usepackage{graphicx}
\definecolor{lightgrey}{RGB}{170, 170, 170}
\definecolor{darkblue}{RGB}{0, 0, 0}
\definecolor{darkred}{RGB}{170, 0, 0}
\definecolor{darkgreen}{RGB}{0, 110, 0}

% Acronyms
\usepackage{xspace}
\newcommand{\sparql}{{SPARQL}\xspace}
\newcommand{\sparqlo}{{SPARQL 1.1}\xspace}
\newcommand{\arq}{{ARQ}\xspace}
\newcommand{\wthreec}{{W\oldstylenums 3C}\xspace}
\newcommand{\sfive}{{S\oldstylenums 5}\xspace}
\newcommand{\select}{{SELECT}\xspace}
\newcommand{\construct}{{CONSTRUCT}\xspace}
\newcommand{\ask}{{ASK}\xspace}
\newcommand{\describe}{{DESCRIBE}\xspace}
\newcommand{\from}{{FROM}\xspace}
\newcommand{\odbc}{{odbc}\xspace}

% Tight lists
\usepackage{enumitem}
\setlist{nolistsep}

% Listings and Verbatim environment
\usepackage{fancyvrb}
\usepackage{relsize}
\usepackage{listings}
\usepackage{verbatim}
\newcommand{\smalllistingsize}{\fontsize{8pt}{9.5pt}}
\newcommand{\inlinelistingsize}{\fontsize{8.5pt}{11pt}}
\newcommand{\defaultlistingsize}{\inlinelistingsize}
\RecustomVerbatimCommand{\Verb}{Verb}{fontsize=\inlinelistingsize}
\RecustomVerbatimEnvironment{Verbatim}{Verbatim}{fontsize=\inlinelistingsize}
\lstset{frame=lines,captionpos=b,numberbychapter=false,escapechar=§,
        belowskip=1em,
        xleftmargin=2ex,
        framexleftmargin=2ex,
        basicstyle=\ttfamily\smalllistingsize\selectfont}
\crefname{lstlisting}{Listing}{Listings}
\definecolor{grey}{RGB}{130,130,130}

\usepackage{color}
\newcommand{\todo}[1]{\noindent\textcolor{red}{{\bf \{TODO}: #1{\bf \}}}}
\begin{document}

\title{What we can learn from one year of\\ public transit route planning API queries}
\numberofauthors{6}
\author{
\alignauthor
Pieter Colpaert\\
 \affaddr{\email{\texttt{pieter.colpaert@ugent.be}}}
\and
\alignauthor
Alvin Chua\\
 \affaddr{\email{\texttt{alvin.chua@asro.kuleuven.be}}}
\and
\alignauthor
Ruben Verborgh\\
 \affaddr{\email{\texttt{ruben.verborg@ugent.be}}}
\and
\alignauthor
Erik Mannens\\
 \affaddr{\email{\texttt{erik.mannens@ugent.be}}}
\and
\alignauthor
Rik Van de Walle\\
 \affaddr{\email{\texttt{rik.vandewalle@ugent.be}}}
\and
\alignauthor
Andrew Vande Moere\\
 \affaddr{\email{\texttt{andrew.vandemoere@asro.kuleuven.be}}}\\
}

\maketitle
\begin{abstract}
% Context
%Public transit schedules are made available on the Web in various ways, such as in a GTFS file, a route planning API, or through Linked Connections.
The iRail project hosts a route planning API for the Belgian railway company.
This webservice provides a concise answer to a complex question with parameters such as ``departuretime'', ``from'' and ``to''.
% Need
In the field of urban planning, researchers need an indication of how people move between cities. 
Yet, getting this data from official sources has proven to be troublesome.
% Task
In this paper, we analyse the query logs from the iRail API, to tell us something about how people move between cities in Belgium.
% Object
We studied queries for the year 2013, which contains an average of 3000 queries per day.
% Findings
We found X, Y, Z interesting patterns (\todo{Alvin}), which correspond with reality.
% Conclusion
This proves that query logs from route planning systems are valuable as an indication of transit flows and may house interesting stories.
% Perspectives
However, many private owned route planners exists, which keep these query logs strictly confidential.
Bearing this in mind, we discuss the current state of the art in public transit schedule publishing, and formulate opportunities for gathering query logs when using a Linked Data Fragments approach.

\end{abstract}

\vspace{1em}

\section{Introduction}
\label{sec:introduction}

%What's iRail

\section{Gathering the logs}
\label{sec:logs}

The logs are available at \ldots

\section{Looking for patterns}
\label{sec:method}


\section{Results}
\label{sec:results}

\todo{add a couple of nice images with explanation here}

\subsection{Pattern 1: ...}
\label{sec:pattern1}


\subsection{Pattern 2: ...}
\label{sec:pattern2}

\subsection{Pattern 3: ...}
\label{sec:pattern3}

\section{Publishing transport data}

Thoughts on publishing transport data on the web and query logs in the case of GTFS, Linked Connections\cite{lc} (linked data fragments\cite{ldf}) and webservices.

\section{Conclusion}
\label{sec:conclusion}

Query logs are very interesting

% Fix spacing after References header (as line 1308 of sig-alternate.cls breaks it)
\let\oldsection\section
\renewcommand{\section}[2][1]{\oldsection{#1}\vspace{-3pt}}

\bibliographystyle{abbrv}
\bibliography{refs}
\end{document}
