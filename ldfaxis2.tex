
% Linked Data Fragments axis
\begin{figure}[t]
  \begin{tikzpicture}[
    x=.005\textwidth,
    y=.1cm,
    axis/.style={
      line width=.8pt,
      stealth-stealth,
      color=black!65,
    },
    axislabel/.style={
      inner sep=0,
      font=\slshape\fontsize{8}{9}\selectfont,
      color=black!70,
    },
    tick/.style={
      line width=1pt,
    },
    ticklabel/.style={
      anchor=south,
      inner sep=0,
      align=center,
      font=\bfseries\fontsize{8}{8}\selectfont,
      text depth=0pt,
    },
  ]
    \newcommand\tick[2]{%
      \draw[tick] (#1, 1.7) -- (#1, -.4pt);
      \node[ticklabel] at (#1, 2.4) {#2};
    }
    \newcommand\vaguetick[1]{%
      \draw[tick,densely dotted] (#1, 1.7) -- (#1, -.4pt);
    }

    \draw[axis] (-100, 0) -- (-5, 0);
    \node[axislabel,align=left,anchor=north west] at (-100, -2.3)
    {
%      generic requests\\
      high client effort\\
      high server availability\\
      know how many times datadump was downloaded
    };
    \node[axislabel,align=right,anchor=north east] at (-5, -2.3)
    {
%      specific requests\\
      high server effort\\
      low server availability\\
      detailed query logs
    };
    %\node[text,ticklabel] at (-53, 2.4) {Linked\\Connections};
    \tick{-90}{GTFS data\\dump}
    \tick{-15}{route planning\\service}
    \vaguetick{-23}
    \vaguetick{-32}
    \vaguetick{-41}
    \vaguetick{-50}
    \vaguetick{-60}
    \vaguetick{-70}
    \vaguetick{-80}

   %\node[align=center,font=\fontsize{8}{9}\selectfont\bfseries] at (0, -8.3)
   %   {various types of\\Linked Data Fragments};
  \end{tikzpicture}
  \vspace{0.01pt}% To create a bit of vertical space between the figure captions
  \caption{This axis, first introduced by Linked Data Fragments~\cite{ldf}, illustrates the possibilities within publishing transport data. Different ticks on the axis illustrate client effort versus server expressivity: on the far left, data dumps offer high server availability, yet the effort needed by data reusers is high, and query logs do not reach the server. Moving to the right, we identify Linked Connections~\cite{lc} as an in-between solution. On the far right, route planning services require high server effort, leading to detailed query logs.}
  \label{fig:LDFAxis2}
\end{figure}